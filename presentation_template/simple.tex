\documentclass[8pt,aspectratio=169,xcolor=dvipsnames]{beamer} % use handout option for handouts
%\usepackage{pgfpages}
%\pgfpagesuselayout{2 on 1}[a4paper,border shrink=5mm]
\usepackage[utf8]{inputenc}
\usepackage{colortbl}
\usepackage{multimedia}

\usetheme{default}
\usecolortheme{seagull}

\usepackage[T1]{fontenc}
\usepackage[light]{roboto} 

\usepackage{mwe} % provides images used in this example

%%%%%%%%%%%%%%%%%%%%%%%%%%%%%%%%%%%%%%%%%%%%%%%%%%%%%%%%%%%%%%%%%%%%%%%%%%%%%%%%%%%%%%%

\setbeamertemplate{frametitle}{\color{black}\bfseries\vskip16pt\insertframetitle}
\setbeamertemplate{footline}[frame number]

\linespread{1.25}

\definecolor{ImperialRed}{HTML}{e63946}
\definecolor{HoneyDew}{HTML}{f2efef}
\definecolor{PowderBlue}{HTML}{A8DADC}
\definecolor{CeladonBlue}{HTML}{457B9D}
\definecolor{PrussianBlue}{HTML}{1D3557}


\definecolor{UBCgrey}{rgb}{0.3686, 0.5255, 0.6235} % UBC Grey (secondary)

\setbeamercolor{block title}{bg=PrussianBlue,fg=white}
\setbeamercolor{block title example}{bg=CeladonBlue,fg=white}
\setbeamercolor{block title alerted}{bg=ImperialRed,fg=white}

\setbeamercolor{block body}{bg=HoneyDew,fg=black}
\setbeamercolor{block body example}{bg=HoneyDew,fg=black}
\setbeamercolor{block body alerted}{bg=HoneyDew,fg=black}

\setbeamerfont{block title}{size=\normalsize}
\setbeamerfont{block body}{size=\normalsize}
\setbeamerfont{block title example}{size=\normalsize}
\setbeamerfont{block body alerted}{size=\normalsize}
\setbeamerfont{block body example}{size=\normalsize}

\setbeamercolor{alerted text}{fg=ImperialRed}

% table definitions
\newcolumntype{a}{>{\columncolor{ImperialRed!70}}c}
% Change horizontal spacing
\setlength{\tabcolsep}{20pt}
% Change vertical spacing
\renewcommand{\arraystretch}{1.5}


%------------------------------------------------------------
%This block of code defines the information to appear in the
%Title page
\title[About Beamer] %optional
{\textbf{About the Beamer class in presentation making}}

\subtitle{A short story}

\author[Schramm, Georg] % (optional)
{Georg Schramm\inst{1}}

\institute[KUL] % (optional)
{
  \inst{1}%
  Department of Imaging and Pathology\\
  KU Leuven
}

\date[VLC 2021] % (optional)
{Very Large Conference, April 2021}

%\logo{\includegraphics[height=1cm]{overleaf-logo}}

%End of title page configuration block
%------------------------------------------------------------



%------------------------------------------------------------
%The next block of commands puts the table of contents at the 
%beginning of each section and highlights the current section:

\AtBeginSection[]
{
  \begin{frame}
    \frametitle{Table of Contents}
    \tableofcontents[currentsection]
  \end{frame}
}
%------------------------------------------------------------


\begin{document}

%The next statement creates the title page.
\frame{\titlepage}


%---------------------------------------------------------
%This block of code is for the table of contents after
%the title page
\begin{frame}
\frametitle{Table of Contents}
\tableofcontents
\end{frame}
%---------------------------------------------------------


\section{First section}

%---------------------------------------------------------
%Changing visivility of the text
\begin{frame}
\frametitle{Sample frame title}
This is a text in second frame. For the sake of showing an example.

\begin{itemize}
    \item<1-> Text visible on slide 1
    \item<2-> Text visible on slide 2
    \item<3>  Text visible on slide 3
    \item<4-> Text visible on slide 4
\end{itemize}
\end{frame}

%---------------------------------------------------------
\begin{frame}{Colored tables}
\centering
\begin{tabular}{lrara}
\hline
\rowcolor{PowderBlue}
    \textbf{variables} & \multicolumn{4}{c}{\textbf{data}}\\ \hline
    var 1 & 1 & 2 & 3 & 4 \\ \hline
    var 2 & 1 & 2 & 3 & 4 \\ \hline
    var 3 & 1 & 2 & 3 & 4 \\ \hline
\end{tabular}
\end{frame}
%---------------------------------------------------------

\section{Second section}

%---------------------------------------------------------
%Highlighting text
\begin{frame}
\frametitle{Sample frame title}

In this slide, some important text will be
\alert{highlighted} because it's important.
Please, don't abuse it.

\begin{block}{Remark}
Sample text
\end{block}

\begin{alertblock}{Important theorem}
Sample text in red box
\end{alertblock}

\begin{examples}
Sample text in green box. The title of the block is ``Examples".
\end{examples}
\end{frame}
%---------------------------------------------------------


%---------------------------------------------------------
%Two columns
\begin{frame}
\frametitle{Two-column slide}

\begin{columns}

\column{0.3\textwidth}
This is a text in first column.
$$E=\int f(x) dx$$
\begin{itemize}
\item First item
\item Second item
\end{itemize}

\pause

\column{0.7\textwidth}

\begin{figure}
\begin{center}
  \includegraphics[width=.8\columnwidth]{image-a}
\end{center}
\caption{Foo bar}
\end{figure}

\end{columns}
\end{frame}
%---------------------------------------------------------

%---------------------------------------------------------
%Two columns
\begin{frame}
\frametitle{Math}

\begin{equation}
    \lambda^+ = \lambda \sum_i \frac{y_i}{\bar{y}_i}
\end{equation}

\end{frame}
%---------------------------------------------------------

\begin{frame}[t]
\frametitle{Test}

\begin{block}<1->{Correction method}
  \begin{itemize}
    \item<2-> Simulate
    \item<5-> Assign
  \end{itemize}
\end{block}

\begin{onlyenv}<3-6>
  \begin{center}
    \includegraphics<3>[scale=0.4]{image-a}
    \includegraphics<4>[scale=0.4]{image-b}
    \includegraphics<6>[scale=0.4]{image-c}
  \end{center}
\end{onlyenv}

\begin{onlyenv}<7->
  \begin{block}<7->{Scores}
    \begin{itemize}
      \item<8-> PSNR
      \item<9-> MSE
    \end{itemize}
  \end{block}
\end{onlyenv}

\begin{center}
  \includegraphics<10>[scale=0.2]{image-a}
  \includegraphics<10>[scale=0.2]{image-b}
\end{center}
\end{frame}  

\end{document}
